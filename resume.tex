\documentclass{resume}
\setauthor{Newton Ni}
\setphone{248--686--8887}
\setemail{cn279@cornell.edu}
\setsite{nwtnni.me}

\usepackage{hyperref}

\begin{document}
\begin{education}
    \entry{Cornell University}{College of Engineering}{May 2019}
    \begin{description}
        \item B.S. in Computer Science --- 3.95 GPA
        \item Coursework ---
            Compilers,
            Programming Languages,
            Functional Programming,
            Algorithms,

            Computer Systems,
            Computer Graphics,
            Distributed Systems
    \end{description}
\end{education}

\begin{experience}
    \entry{Foster Lab}{Cornell Dept.\ of Computer Science}{May 2018 --- Current}
    \begin{description}
        \item Design and implement type system for the P4 network programming language
        \item Translate informal P4--16 specification into OCaml logic
    \end{description}

    \entry{Teaching Assistant}{Functional Programming and Data Structures}{Jan 2018 --- Current}
    \begin{description}
        \item Lead semiweekly lecture and exercise-based recitation of 30 students
        \item Create review exercises on concepts like monads, interpreters, and streams
        \item Received average rating of 4.7/5.0 across 19 metrics and 21 student evaluations
    \end{description}

    \entry{Teaching Assistant}{Honors Object-Oriented Programming}{Aug 2017 --- Dec 2017}
    \begin{description}
        \item Held office hours for 10--20 students, one and a half hours per week
        \item Taught lab with four other consultants for 25--35 students, one hour per week
        \item Developed automatic server-based submission format checker
    \end{description}

    \entry{Clark Lab}{Cornell Dept.\ of Molecular Biology and Genetics}{May 2017 --- Oct 2017}
    \begin{description}
        \item Predicted coronary heart disease using random forest classifiers
        \item Extracted predictors from National Heart, Lung, and Blood Institute genome data
        \item Analyzed genotypes using PLINK whole genome association analysis software
    \end{description}
\end{experience}

\begin{projects}
    \entry{tigerc}{\href{https://github.com/nwtnni/tigerc}{Tiger Programming Language Compiler}}{Jun 2018 --- Aug 2018}
    \begin{description}
        \item Compiles high-level Tiger language down to x86--64 assembly using Rust
        \item Performs type-checking, IR translation, naive register allocation, etc.
        \item Applies macro-based metaprogramming for test boilerplate generation
    \end{description}

    \entry{paxos}{\href{https://github.com/nwtnni/paxos}{Paxos Distributed Consensus Protocol}}{Nov 2018 --- Dec 2018}
    \begin{description}
        \item Implements a generic replicated state machine library backed by Multi-Paxos
        \item Verifies correctness with a JSON DSL-based test harness and extensive logging
        \item Includes an example chatroom state machine with runnable server and client
    \end{description}

    \entry{gnocchi}{\href{https://github.com/nwtnni/gnocchi}{Basic Procedurally Generated World}}{Nov 2018 --- Dec 2018}
    \begin{description}
        \item Renders with fog of war and directional shading using JavaScript and WebGL
        \item Supports multiple concurrent players using a client-server architecture
        \item Optimizes performance with chunking, bitflags, and occlusion culling
    \end{description}
\end{projects}

\begin{skills}
    \begin{description}
        \item \textbf{Languages:} Rust, Java, OCaml, Python, Javascript, C
        \item \textbf{Software:} Git, LaTeX, Bash, Unix, Vim, tmux
        \item \textbf{Interests:} Violin, guitar, cooking, reading, technical blogging
    \end{description}
\end{skills}
\end{document}
