\documentclass{resume}
\setauthor{Yuet Ming Leung}
\setphone{607--379--4576}
\setemail{yl787@cornell.edu}
\setsite{github.com/attilusleung}

\usepackage{hyperref}

\begin{document}
\begin{education}
    \entry{Cornell University}{College of Arts and Sciences}{Expected May 2022}
    \begin{description}
        \item Candidate for B.A. in Computer Science --- 4.0 GPA
        % \item Coursework ---
        %     Compilers,
        %     Programming Languages,
        %     Functional Programming,
        %     Algorithms,

        %     Computer Systems,
        %     Computer Graphics,
        %     Distributed Systems
    \end{description}
    \entry{Diocesan Boys' School}{International Baccalaureate Diploma Program}{May 2018}
    \begin{description}
        \item Graduated with Billingual IB Diploma
        \item Courses
            \subitem Higher Level: Physics, Chemistry, Mathematics

            \subitem Standard Level: Economics, Chinese Literature, English Language and Literature
    \end{description}
\end{education}

\begin{experience}
    \begin{entryleft}{Cornell University Autonomous Underwater Vehicle}
        \begin{description}
            \item Created vision modules for the underwater vehicle using OpenCV
            \item Programmed autonomous missions for the vehicle using a custom mission system
            \item Created a new automated testing software for vision modules
            \item Finalist in AUVSI's 2019 RoboSub competition, where the vehicles autonously completed gate navigation, pinger tracking and buoy ramming tasks
        \end{description}
    \end{entryleft}
    \begin{entryright}
        \entryitem{Software Co-Lead}{Aug 2019 --- Current}
        \entryitem{Software Member}{Oct 2018 --- Aug 2019}
    \end{entryright}

    \begin{entryleft}{Diocesan Boys' School Design and Technology Engineering Team}
        \begin{description}
            \item Collaborated with a team of around 10 people
            \item Oversaw the software components of engineering projects
            \item Designed an automated liquid cooling system for a self cooling smart glass
            \item First place in HKUST's Paper Tower Challenge
        \end{description}
    \end{entryleft}
    \begin{entryright}
        \entryitem{Programming Lead}{Dec 2016 --- May 2018}
    \end{entryright}

    \begin{entryleft}{Cornell Computing and Information Science: \\ Introduction to Computing Using Python}
        \begin{description}
            \item Teach two labs per week and hold additional consultant office hours
            \item Grade assignments, preliminary exams and finals
        \end{description}
    \end{entryleft}
    \begin{entryright}
        \entryitem{Consultant}{August 2019 --- Current}
    \end{entryright}
\end{experience}

\begin{projects}
    \entry{tigerc}{\href{https://github.com/nwtnni/tigerc}{tiger programming language compiler}}{jun 2018 --- aug 2018}
    \begin{description}
        \item compiles high-level tiger language down to x86--64 assembly using rust
        \item performs type-checking, ir translation, naive register allocation, etc.
        \item applies macro-based metaprogramming for test boilerplate generation
    \end{description}

    \entry{paxos}{\href{https://github.com/nwtnni/paxos}{paxos distributed consensus protocol}}{nov 2018 --- dec 2018}
    \begin{description}
        \item implements a generic replicated state machine library backed by multi-paxos
        \item verifies correctness with a json dsl-based test harness and extensive logging
        \item includes an example chatroom state machine with runnable server and client
    \end{description}

    \entry{gnocchi}{\href{https://github.com/nwtnni/gnocchi}{basic procedurally generated world}}{nov 2018 --- dec 2018}
    \begin{description}
        \item renders with fog of war and directional shading using javascript and webgl
        \item supports multiple concurrent players using a client-server architecture
        \item optimizes performance with chunking, bitflags, and occlusion culling
    \end{description}
\end{projects}

\begin{skills}
    \begin{description}
        \item \textbf{languages:} python, java, C\#
        \item \textbf{software:} OpenCV, git, latex, bash, unix, vim, docker, flask
    \end{description}
\end{skills}
\end{document}
